\documentclass[english, a4paper, 11pt]{article}
\usepackage[T1]{fontenc}
\usepackage[latin9]{inputenc}
\usepackage{geometry}
\usepackage{csquotes}
\usepackage[style=apa,sortcites=true,sorting=nyt]{biblatex}
\addbibresource{references.bib}
\geometry{verbose,tmargin=3cm,bmargin=3cm,lmargin=3cm,rmargin=3cm}

\makeatletter
\usepackage{url}

\makeatother

\usepackage{babel}


\title{Research Plan Week 1--2}
\author{Thijs Raymakers}
\date{2020-04-20}

\newcommand{\namelistlabel}[1]{\mbox{#1}\hfil}
\newenvironment{namelist}[1]{%1
\begin{list}{}
    {
        \let\makelabel\namelistlabel
        \settowidth{\labelwidth}{#1}
        \setlength{\leftmargin}{1.1\labelwidth}
    }
  }{%1
\end{list}}

\begin{document}
\maketitle

\begin{namelist}{xxxxxxxxxxxxxxxxxxxxxxxxxxxxxxxxxxxxxxx}
\item[{\bf Title:}]
    Gender bias in online communities
\item[{\bf Author:}]
    Thijs Raymakers (4647610)
\item[{\bf Responsible Professor:}]
    David Tax, Marco Loog
%\item[{\bf (Optionally) Other Supervisor:}]
%	Eva
%\item[{\bf (Required for final version) Examiner:}]
%	Another Professor (\emph{interested, but not involved})
\item[{\bf Peer group members:}]
    Pia Keukeleire, 
    Thomas van Tussenbroek,
    David Happel,
    Dina Chen,
    Katja Schmahl
\end{namelist}


\section*{Background of the research}
With the help of the recent \#MeToo movement, the issue of gender discrimination and 
gender biasing got serious attention on social and traditional media. In this research,
I want to look at the gender bias that occurs in written text with the help of natural
language processing and I want to focus on the differences that occur between different
online communities.
This would be interesting to look at, because it could provide a concrete answer to the
question whether online communities with a male majority of contributors are more biased 
towards males and whether online communities with a female majority of contributes are 
more biased towards females. If these questions would both be true, it could indicate that 
a community with an equal amount of male and female contributes would have a smaller
gender bias.

There has been previous research into gender stereotyping between different subreddits by \cite{kohn_2018_gender} with the help of word embeddings.  While \cite{kohn_2018_gender}
concludes that there is a statistically significant gender bias between certain different 
topics on Reddit, it is also acknowledged that the used method might have some weaknesses.

I am certain that there has been more research in this area, but I was not able to find
more relevant literature for other public communities, like Twitter. This is an area that
can still be explored and could possible be explored within this research.


\section*{Research Question}
This research would explore the proposed research question RQ2, where gender bias for different Reddit subreddits or different Twitter hashtags is measured. Whether this research
would focus on Reddit subreddits or on Twitter hashtags, is still something that should be
explored and something that is depended on factors like the availability and usability of
the datasets. This exploratory research can be performed in the first weeks and the
research question will be changed based upon the initial research.

The hypothesis will be formulated along the research question after initial research into
the two possible approaches, either Reddit subreddits or Twitter hashtags.

\section*{Method}
This research will most likely create word embeddings for different communities. This can
done either with something like word2vec or fastText. The gender bias itself could be
measured with methods described by \cite{bolukbasi_2016_quantifying_stereotypes} or 
\cite{caliskan_2017_semantics_language_corpora}.

An unanswered question that still remains is the question about the data. To what extend
is the data of Twitter and/or Reddit usable for this kind of research and how would it be
obtained. This is something that has to be explored in the first weeks.


\section*{Planning of the first two weeks}
The first two weeks of the research project will be filled with a more detailed research
into the exact methods and the availability and usability of the datasets. The latter is
especially important, because it is crucial in the decision whether the research will
focus on Twitter hashtags or Reddit communities.

The following two will also be used to talk with peer group members that will also work
on word2vec or fastText, to discuss our thoughts about either approach. It will also
be used to discuss different ways of measuring gender bias in word embeddings.

After these first two weeks, the research question and the research proposal, will be
more detailed and will provide more details about my approach.

\printbibliography

\end{document}
