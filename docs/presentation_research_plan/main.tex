% We ask you to update your research plan if you haven't done so already with our feedback. Please prepare a short powerpoint talk (max 5 minutes / 5 slides per student) describing your research plan. Include: 
%- a brief motivation / context
%- your research question
%- and how you plan to answer it (algorithms, code, etc.).
%- a planning (perhaps a table with plans per week)

\documentclass{beamer}
\usepackage[english]{babel}
\usepackage[backend=biber,style=authoryear,sorting=none]{biblatex}
\addbibresource{references.bib}
\usetheme{Madrid}

% Cite with brackets
\newcommand{\bcite}[1]{(\cite{#1})}

\author{Thijs Raymakers}
\title{Presentation research plan}
\date{April 29, 2020}

\begin{document}

\section{Context}
\begin{frame}
\frametitle{Context}
\begin{itemize}
    \item Gender bias is present in word embeddings~\bcite{bolukbasi_2016_quantifying_stereotypes}
    \pause
    \item Typically measured the help of analogies
    \begin{itemize}
        \item \textit{man} is to \textit{king} as \textit{woman} is to \textit{queen}
        \item \textit{man} is to \textit{doctor} as \textit{woman} is to \textit{nurse}~\bcite{bolukbasi_2016_quantifying_stereotypes}
    \end{itemize}
    \pause
    \item However, analogies might not be the right tool to measure bias \bcite{2019arXiv190509866N}
    \begin{itemize}
        \item Implementations of analogies are too restrictive
        \item Variables in the analogy are subjective choices
    \end{itemize}
\end{itemize}
\end{frame}

\section{Research question}
\begin{frame}
\frametitle{Research question}

To what extend differ word embeddings trained on different languages when comparing
the words man and woman?


\end{frame}

\section{Method}
\begin{frame}
\frametitle{Method}
\end{frame}

\section{Planning}
\begin{frame}
\frametitle{Planning}
\end{frame}

\section{References}
\begin{frame}[allowframebreaks]
\frametitle{References}
\printbibliography[heading=none]
\end{frame}

\end{document}

