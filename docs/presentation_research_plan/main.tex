% We ask you to update your research plan if you haven't done so already with our feedback. Please prepare a short powerpoint talk (max 5 minutes / 5 slides per student) describing your research plan. Include: 
%- a brief motivation / context
%- your research question
%- and how you plan to answer it (algorithms, code, etc.).
%- a planning (perhaps a table with plans per week)

\documentclass{beamer}
\usepackage[english]{babel}
\usepackage{csquotes}
\usepackage[backend=biber,style=apa,sorting=none]{biblatex}
\addbibresource{references.bib}
\usetheme{Madrid}

\author{Thijs Raymakers}
\title{Presentation research plan}
\date{April 29, 2020}

\begin{document}

\section{Context}
\begin{frame}
\frametitle{Context}
\begin{itemize}
    \item Gender bias is present in word embeddings~\footfullcite{bolukbasi_2016_quantifying_stereotypes}
    \pause
    \item Interesting to look at different languages
    \pause
    \item Typically measured the help of analogies
    \begin{itemize}
        \item \textit{man} is to \textit{king} as \textit{woman} is to \textit{queen}
        \item \textit{man} is to \textit{doctor} as \textit{woman} is to \textit{nurse}$^1$
    \end{itemize}
    \pause
    \item However, analogies might not be the right tool to measure bias \footfullcite{2019arXiv190509866N}
    \begin{itemize}
        \item Implementations of analogies are too restrictive
        \item Variables in the analogy are subjective choices
    \end{itemize}
\end{itemize}
\end{frame}

\section{Research question}
\begin{frame}
\frametitle{Research question}
To address the issues raised by \textcite{2019arXiv190509866N}, the proposed method
and the research question do not directly measure gender bias.
\newline
\pause
\begin{block}{Research question}
To what extent differ word embeddings trained on different languages when comparing
the words man and woman?
\end{block}
\end{frame}

\section{Method}
\begin{frame}
\frametitle{Method}
\begin{itemize}
    \item Use fastText word embedding\footfullcite{grave2018learning}
    \pause
    \item Still based on analogies
    \pause
    \item To address the restrictiveness
    \begin{itemize}
        \item Unconstrained\footfullcite{2019arXiv190509866N} variant of \texttt{3CosMul} algorithm \footfullcite{Levy14linguisticregularities}
        \item Less accurate, but not constrained
    \end{itemize}
    \pause
    \item To address subjectiveness
    \begin{itemize}
        \item Keep two variables fixed, two variables free
            \newline
        (\textit{man} is to X as \textit{woman} is to Y)
    \end{itemize}

\end{itemize}
\end{frame}

\begin{frame}
\frametitle{Method}
\begin{block}{Ratio}
$\forall x$ in embedding $W$, count the number of times where ``\textit{man} is to $x$ as \textit{woman} is to $y$" and ``\textit{woman} is to $x$ as \textit{man} is to $y$" such that $x=y$
\end{block}
Can be used to calculate ratio of analogy "equalness". This ratio can be used to
compare different word embeddings.

\end{frame}

\section{Planning}
\begin{frame}
\frametitle{Planning}
\begin{tabular}{| l | p{10cm} |}
    \hline
    Week & Plans \\ \hline
    2 & Write introduction \& work on baseline \\
    3 & Write related work \& improve on baseline \\
    4 & Improve introduction \& improve baseline \\
    5 & Prepare presentation \& write evaluation \& describe and improve baseline \\
    6 & Write evalutation and conclusion \\
    7 & Submit first draft and improve using feedback \\
    8 & Submit second draft and improve using feedback \\
    9 & Improve using feedback and submit final paper \\
    10 & Prepare poster presentation \\
    \hline
\end{tabular}
\end{frame}

\section{References}
\begin{frame}
\frametitle{References}
\printbibliography[heading=none]
\end{frame}

\end{document}

