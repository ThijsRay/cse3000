\section{Conclusion}
This study set out to assess to what extent word embeddings of different
languages are biased towards gender. The results of this investigation show
that word embeddings of different languages have a bias towards gender and
that they are generally more biased towards \emph{male} than towards \emph{female}.
The insight gained from
this contribution may be of assistance to future debiasing efforts and possibly
affirms the linguistic relativity hypothesis~\parencite{lucy_linguistic_1997}.
In spite of the limitations of this study, the study certainly adds to our understanding
of gender bias in word embeddings of different languages.

Future research is needed to fully understand why certain languages seem to be
more biased towards \emph{male} or \emph{female}. Further work needs to be done to
establish why the majority of the tested languages have a bias towards \emph{male}.
Finally, it is recommended that this research is replicated for different forms
of bias, such as racial or sexuality bias.

\section{Acknowledgements}
I would like to thank my supervisors for the help and expertise they have provided
during the course of this project. I would also like to thank my peers who were able
to give me feedback during the process.
