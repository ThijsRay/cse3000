\section{Introduction}
Word embeddings have become an important tool in the field of natural language
processing. They can be used to represent the semantics and the meaning of
words as a vector of numbers, which is used for all sorts of applications, like
recommendation systems~\parencite{10.1145/3219819.3219885} or résumé
scanning~\parencite{large-scale-hoang-2017}. Word
embeddings are often used to solve analogies. An example of this would be that the
vector that represents the word
$\overrightarrow{queen}$ would be equal to $\overrightarrow{king} -
\overrightarrow{man} +
\overrightarrow{woman}$. This analogy would capture an `opposite gender' relation.
Other examples of these analogies would be $\overrightarrow{Paris} -
\overrightarrow{France} + \overrightarrow{Germany} \approx \overrightarrow{Berlin}$,
capturing a `capital of' relation and
$\overrightarrow{cars} - \overrightarrow{car} + \overrightarrow{apple} \approx
\overrightarrow{apples}$, which would capture a `pluralisation' relation~\parencite{vylomova-etal-2016-take}.
However, these word embeddings should be used with care because they have been
shown to contain human
biases~\parencite{caliskan_2017_semantics_language_corpora}.
\textcite{caliskan_2017_semantics_language_corpora} showed
that words related to science are more related to male terms and that
words related to arts are more related to female terms. This is a form of gender bias
that can be problematic when the word embedding is used in a scenario where gender bias
should be avoided, such as
the case in which Amazons recruiting tool penalized female candidates, because they
included words related to women on their résumé~\parencite{dastin_2018}.

Recently, efforts have been made to reduce the effect of bias in word embeddings.
\textcite{bolukbasi_2016_quantifying_stereotypes} showed that debiasing word embeddings
is possible to some extend, but \textcite{gonen-goldberg-2019-lipstick-pig} showed that
the approach of \textcite{bolukbasi_2016_quantifying_stereotypes} merely hides the
gender bias, instead of removing it.

This paper aims to take a step back and look into whether there is a difference in gender
bias between word embeddings of different languages. Generally, debiasing efforts
have focussed on word embeddings of the English language. However, it might be the
case that some languages are less biased than the English language. Knowing whether this
is the case can enable future research to look into why certain languages are or are not
more gender biased than others. This insight could be useful in the search towards a novel
debiasing algorithm.
