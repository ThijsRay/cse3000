% 1) the overall purpose of the study and the research problem(s) you investigated; 
% 2) the basic design of the study; 
% 3) major findings or trends found as a result of your analysis; 
% 4) a brief summary of your interpretations and conclusions.

% 1) the context or background information for your research; the general topic under study; the specific topic of your research
Word embeddings are useful for various applications, such as sentiment
classification~\parencite{tang2014learning}, word translation~\parencite{xing2015normalized} and résumé parsing~\parencite{nasser2018convolutional}. 
% 3) what’s already known about this question, what previous research has done or shown
Previous research has determined that word embeddings contain gender
bias, which can be problematic in certain applications such as résumé parsing.
% 2) the central questions or statement of the problem your research addresses
This research has addressed the question whether gender bias is present in word
embeddings of different languages.
% 4) the main reason(s), the exigency, the rationale, the goals for your research—Why is it important to address these questions? Are you, for example, examining a new topic? Why is that topic worth examining? Are you filling a gap in previous research? Applying new methods to take a fresh look at existing ideas or data? Resolving a dispute within the literature in your field? . . .
%Future research can use this knowledge to look for novel debiasing techniques based on
%characteristics of languages that show different behaviour when bias is measured.
% 5) your research and/or analytical methods
Gender bias has been measured on word embedding of 26 different languages with the help
of the Word Embedding Association Test
by~\textcite{caliskan_2017_semantics_language_corpora}.
% 6) your main findings, results, or arguments
The results show that most of the tested languages seem to have bias towards male,
while a few languages seem to have a bias towards female.
% 7) the significance or implications of your findings or arguments.
This result is surprising and somewhat contradicts previous literature.
\newline
\newline
\emph{Keywords}: natural language processing, gender bias, word embedding, WEAT, language
