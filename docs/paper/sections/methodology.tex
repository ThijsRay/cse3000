\section{Methodology}
A common way of measuring gender bias in word embeddings uses the 
WEAT~\parencite{caliskan_2017_semantics_language_corpora} method. This method measure the
association between a set of \textit{attribute words} (e.g. \textit{man}, \textit{woman}) 
and a set of \textit{target} words (e.g. \textit{programmer}, \textit{family}, \textit{nurse}, \textit{engineer}). This
approach can work when looking at a single language, but has some problems when 
trying to compare differences between
different languages. 

First, not all words have a one-to-one translation into another
language, because certain words might have a different meaning, depending on the
context in which the word is used. It is therefore not possible to translate all the
attribute and target words into a different language, because it might lead to an
incorrect and unfair comparison. A great example of this would the English word
\textit{man}, which can mean \textit{male} or \textit{mankind}, depending on the context.
Second, the inclusion or exclusion of words from
the target set is a subjective decision~\parencite{nissim_fair_is_better_2020}. Therefore,
the measurement itself could be biased, depending on what words are or are not in the set of target words.

The first problem can be addressed by limiting the amount of words that have to be
translated. The method that is used in this research will use the two attribute words 
\textit{male} and \textit{female}. The assumption is made that these words are universal
and translatable in all languages. However, this is an assumption that is difficult to
verify without having a good understanding of all the languages that are involved.

The second problem can be addressed in two ways. The goal is to create a reproducible
set of target words for all languages. This can be done by either looking at all words
in the
word embedding, or by looking at the most used words in the word embedding.

The first way uses the
all the words in the word embedding as target words. The advantage of this approach is 
that all words are considered, which eliminates the inclusion or exclusion subjectiveness
described by \textcite{nissim_fair_is_better_2020}. A disadvantage of this approach
are that the results are difficult to interpret, because words with an explicit
gender (e.g. \textit{waiter}, \textit{waitress}, \textit{king}, \textit{queen}) are also 
present.

The second way still uses all the words in the word embedding, but attaches a weight to
each word, depending on its frequency. Effectively, this results in a measurement over
the most commonly used words, because those words have the highest weight.
This approach builds upon the assumption that words that are regarded as neutral words
(e.g. \textit{and}, \textit{the}) should be gender neutral, unless there exists some form
of gender bias in the word embedding.

The advantage of this second approach is that results are more representative of how
the language is actually used. While the amount of words associated with men and the
amount of words associated with women might be equal, i.e. each male word has a female
counterpart, they are not used as often.
Taking the frequency of words into account addresses this issue. 
The disadvantage of this is that
it might miss forms of gender bias that might be present in the word embedding, because
not all words are considered to be as important as other words.

Both of these approaches will be explored. Based on the considerations made above, the
test statistic used by
\textcite{caliskan_2017_semantics_language_corpora} can be reused with some adjustments. 
$$
u(w,W,A,B) = \text{mean}_{\forall a \in (A \cap W)}\text{cos}(\overrightarrow{w}, \overrightarrow{a}) - \text{mean}_{\forall b \in (B \cap W)}\text{cos}(\overrightarrow{w}, \overrightarrow{b})
$$
measures the difference between the \textit{cosine similarity} of a word $w$ of language $W$ and the set of attribute words $A$, and of the cosine similarity of a word $w$ and the
set of attribute words $B$. The $A \cap W$ and $B \cap B$ ensure that the cosine
similarity is only measured between two vectors in the same word embedding. The
calculation of this difference can be used in
$$
t(w,W,A,B) = Z(\text{rank}(w), |W|) * u(w,W,A,B)\quad\text{where}
$$
$$
Z(k, N) = 1 \quad\text{or}\quad Z(k, N) = \frac{1/k}{\displaystyle\sum_{n=1}^{N}(1/n)}
$$
This formula adds a weight to each word, depending on its frequency.
In case the first approach is used, where all words are considered, $Z(k,N)$ will have a
fixed value of 1 because the first approach does not include frequency.
In case the second approach is used, the cosine simularity will get a weight attached
to it because it built upon the assumption that
if the word \textit{x} occurs more often than the word \textit{y}, then the former should be given more weight in the calculation, because that would more accurately reflect the usage of the words. The pre-trained models provided by \textcite{grave2018learning} are sorted by frequency, but exact information about how often a word occurs is omitted.
Fortunately, the information about the frequency can be roughly approximated with the
help of Zipf's law~\parencite{Zipf-1935} or $Z(k,N)$ where $k$ is the frequency rank of the
word and where $N$ is the number of words in the language.

This results in a test statistic of
$$
s(X,Y,A,B) = \displaystyle\sum_{x \in X}t(x,X,A,B) - \displaystyle\sum_{y \in Y}t(y,Y,A,B)
$$
where \textit{A} and \textit{B} are two sets of attribute words, in this case the
translated versions of the words \textit{male} and
\textit{female} and where \textit{X} and \textit{Y} are two sets of target words
(e.g. \textit{engineer} and \textit{scientist}), in this case the sets of words from
two distinct languages. 

The one-sided $p$-value of the permutation test is described as
$$
\text{Pr}_i[s(X_i, Y,_i, A, B) > s(X,Y,A,B)]
$$
where $\{(X_i, Y_i)\}_i$ denotes all partitions of $X \cup Y$ into two sets of equal 
size.

The effect size can be described as
$$
\frac{
    \text{mean}_{x \in X} u(x, X, A, B) - \text{mean}_{y \in Y} u(y, Y, A, B)
}{
    \text{std-dev}_{w \in X \cup Y} u(w,W,A,B)
}
$$
where $W$ is the language that $w$ is a part of.


The languages that have been used in this research, have been chosen based on a
few critera,
\begin{seriate}
    \item the language has a pre-trained word embedding from \textcite{grave2018learning}
    \item the language should be machine translatable with the help of tools like DeepL or
    Google Translate and
    \item the language is part of a different language family than the languages that
    were already chosen.
\end{seriate}
This resulted 26 languages from 16 different language families. An overview of these languages, their language families and the translations of \textit{male} and \textit{female}
can be found in table~\ref{appendix:languages}.
