\section{Setup}
The results were calculated on a machine with 10 CPU cores with 512 GB of
memory. The program calculates the results by first downloading
the required pre-trained word embeddings created by \textcite{grave2018learning}.
For each downloaded language, the words \textit{male} and \textit{female} have been
translated beforehand.
The vectors associated with the translated words \textit{male} and \textit{female} are
extracted from the word embedding. Then, the cosine similarity is calculated
according to the methodology.

The code used to calculate the results can be found on
GitHub\footnote{Git repository can be found on \url{https://github.com/ThijsRay/cse3000}}
and was run with Python version 3.8.2 with R version 4.0.0.

\section{Results}
%\begin{figure*}
%    \caption{Mean difference in cosine distance between \emph{male} and \emph{female}
%    by language}
%    \label{fig:mean_difference}
%    \fitfigure{hist_bias.pdf}
%    \figurenote{Figure~\ref{fig:mean_difference} shows the mean of difference in
%        cosine distance between the translation of \emph{male} and \emph{female}. In
%        other words, whether words
%        are, on average, more associated with \emph{male} or more associated with
%        \emph{female}. A value greater than zero indicates that a words are on average more
%        associated with \emph{male}, while a value below zero indicates that words are on
%        average more associated with \emph{female}.
%    }
%\end{figure*}
%
%\begin{figure*}
%    \caption{Sum of weighted cosine distances between \emph{male} and \emph{female}
%    by language}
%    \label{fig:sum_weighted}
%    \fitfigure{hist_wdiff.pdf}
%    \figurenote{Figure~\ref{fig:sum_weighted} shows the sum of the weighted cosine
%        distance between the translation of \emph{male} and \emph{female}. In other
%        words, whether the most used words are more associated with \emph{male} or
%        more associated with \emph{female}.
%        A value greater than zero indicates that a words are more
%        associated with \emph{male}, while a value below zero indicates that words are
%        more associated with \emph{female}.
%}
%\end{figure*}

\begin{figure*}
    \caption{Effect size of the mean difference in cosine distance between \emph{male} and \emph{female}
    by language}
    \label{fig:effect_size}
    \includegraphics[width=\textwidth]{hist_effect_size.pdf}
    \figurenote{Figure~\ref{fig:effect_size} shows the measured effect size of method 1,
        whether words
        are, on average, more associated with \emph{male} or more associated with
        \emph{female}. A value greater than zero indicates that a words are on average more
        associated with \emph{male}, while a value below zero indicates that words are on
        average more associated with \emph{female}.
    }
\end{figure*}

\begin{figure*}
    \caption{Effect size of the sum of weighted cosine distances between \emph{male} and \emph{female}
    by language}
    \label{fig:weffect_size}
    \includegraphics[width=\textwidth]{hist_weffect_size.pdf}
    \figurenote{Figure~\ref{fig:weffect_size} shows the measured effect size of
        method 2, whether the most used words are more associated with \emph{male} or
        more associated with \emph{female}.
        A value greater than zero indicates that a words are more
        associated with \emph{male}, while a value below zero indicates that words are
        more associated with \emph{female}.
        Languages appended with an asterisk were found to have a \emph{p} value > 0.001.
    }
\end{figure*}

\begin{table}
    \centering
    \begin{threeparttable}
        \caption{Calculated \emph{p} values from values method 2}
        \label{tab:p-values-method-2}
        \begin{tabular*}{0.9\columnwidth}{l@{\extracolsep{\fill}}l}
            \hline
            Language & \emph{p} value \\ \hline
            Portuguese & 0.173 \\
            Russian & 0.424 \\
            Japanese & 0.008 \\
            Turkish & 0.033 \\
            Korean & 0.003 \\
            French & 0.146 \\
            Polish & 0.448 \\
            Hungarian & 0.456 \\
            Thai & 0.494 \\
            Javanese & 0.017 \\
            \hline
        \end{tabular*}
        \begin{tablenotes}[para,flushleft]
            {\small \textit{Note:} All languages with \emph{p} value < 0.001 are not
            shown in this table.}
        \end{tablenotes}
    \end{threeparttable}
\end{table}

An approximation of the \emph{p} value has been performed with a random permutation test with $N = 1000$ random permutations. When performed on the set generated by method 1, it
was found that all languages have a \mbox{\emph{p} value < 0.001} with respect to the dummy
language described in the methodology. This was not the case when the random permutation
test was performed on a set generated by method 2, where some languages did have a
\mbox{\emph{p} value > 0.001}. These languages and their respective \emph{p} values
are shown in table~\ref{tab:p-values-method-2}.

In figure~\ref{fig:effect_size} the effect size is shown for each language for method 1.
Finnish has the highest positive effect size of all the tested languages, indicating that
Finnish has the highest association with \emph{male} of all the tested languages.
On the other hand, Basque has the highest negative effect size of all the tested languages,
an indication that Basque has the highest association with \emph{female} of all the tested
languages.
Most of the tested languages have a positive effect size, indicating that most of the
languages have a higher association with \emph{male} than \emph{female} when the
association is calculated with method 1.

In figure~\ref{fig:weffect_size} the effect size is shown of each language for method 2.
Burmese has highest positive effect size of all the tested languages, indicating that
the most used words in Burmese have the highest association with \emph{male} of all the
tested languages.
Greek has the highest negative effect size of all the tested languages, indicating that
the most used words in Greek have the highest association with \emph{female} of all the
tested languages.
Most of the tested languages have a positive effect size, indicating that the most used
words of most of the languages have a higher association with \emph{male} than
\emph{female} when the association is calculated with method 2.
