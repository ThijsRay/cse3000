\section{Discussion}
%There are some limitations with this approach.
%
%- Concept of 'man' and 'woman' might not be properly translated
%- Doesn't give much insight into why
There are some limitations with the used method, mainly that it is built upon quite a few
assumptions.
The assumption that the words attribute `male` and `female` are universal is difficult
to verify without having a good understanding of all the languages that are involved.
This issue can be solved by only looking at the languages that I have a thorough
understanding of, but this would not enable me to look at languages that are completely
different from English.

Another assumption is that the most commonly used words, like \textit{and} and \textit{the}
are gender neutral. This might not necessarily the case, since words like \textit{he} and
\textit{she} are also very common. Attaching a weight based on the frequency might do
nothing more than measure the frequency of gender specific words, because they have a
more pronouced cosine distance.

It is also difficult to make any conclusions about the differences between the languages
themselves, because it is not fair to assume that a word embedding is a perfect
representation of a language. Besides that, it is difficult to assume that this is
the best method of measuring gender bias of a language as a whole. All of the used
test statistics have their limitations, which makes it difficult to draw a clear
conclusion based on the test statistics.
