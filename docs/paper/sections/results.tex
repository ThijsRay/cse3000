\section{Results} \label{ch:results}
The results are divided in two sections. The first section looks into
the results of the uniform weighting method and the second section focuses on 
the frequency weighting method. Each section addresses the results of the
permutation test from formula \eqref{eq:setup_p_test}. We will also take a look
at the effect sizes and means of each method.

\subsection{Uniform weighting method}
The permutation test has been performed on the uniform weighting method and it showed that
all 26 languages have a \mbox{\emph{p} value < $\alpha$} with respect to
the dummy language described in section~\ref{seq:dummy_language}. 

In figure~\ref{fig:effect_size} the effect size is shown for each language for the
uniform weighting method.
Finnish has the highest positive effect size of all the tested languages, indicating that,
on average,
Finnish words have the highest association with \emph{male} of all the tested languages.
On the other hand, Basque has the highest negative effect size of all the tested languages,
showing that Basque words, on average, have the strongest relation with \emph{female} of
all the tested languages.
Most of the tested languages have a positive effect size, which demonstrates that most of
the languages have a stronger link with \emph{male} than \emph{female} when this
is calculated with the uniform weighting method.

Figure~\ref{fig:mean_difference} presents the mean of the cosine distance and it
shows a similar result to figure~\ref{fig:effect_size}. Most of the languages
have a positive value, which leads to the same observation as in
figure~\ref{fig:effect_size}. In this figure, Hungarian has the most 
association with \emph{male} and Hindi has the most association with \emph{female}.

\subsection{Frequency weighting method}
The permutation test has also been performed on the frequency weighting method,
which showed that most languages have a \mbox{\emph{p} value < $\alpha$} with respect to
the dummy language described in section~\ref{seq:dummy_language}. The languages
that had a \mbox{\emph{p} value > $\alpha$} and their respective \emph{p} values
are shown in table~\ref{tab:p-values-frequency}.

Figure~\ref{fig:weffect_size} shows the effect size of each language for the
frequency weighting method.
Burmese has highest positive effect size of all the tested languages, indicating that
the most used words in Burmese have the highest link with \emph{male} of all the
tested languages.
Greek has the highest negative effect size of all the tested languages, indicating that
the most used words in Greek have the strongest relation with \emph{female} of all the
tested languages.

Figure~\ref{fig:sum_weighted} displays the weighted mean of the cosine distance for
the frequency weighting method. Again, the majority of the languages have a positive
value, which leads to the same observation made for figure~\ref{fig:sum_weighted}. 

\begin{table}[H]
    \begin{threeparttable}
        \caption{Calculated \emph{p} values from values of the frequency weighting
        method}
        \label{tab:p-values-frequency}
        \begin{tabular*}{\columnwidth}{l@{\extracolsep{\fill}}l}
            \hline
            Language & \emph{p} value \\ \hline
            Portuguese & 0.173 \\
            Russian & 0.424 \\
            Japanese & 0.008 \\
            Turkish & 0.033 \\
            Korean & 0.003 \\
            French & 0.146 \\
            Polish & 0.448 \\
            Hungarian & 0.456 \\
            Thai & 0.494 \\
            Javanese & 0.017 \\
            \hline
        \end{tabular*}
        \begin{tablenotes}[para,flushleft]
            {\small \textit{Note:} All languages with \emph{p} value < $\alpha$ are not
            shown in this table.}
        \end{tablenotes}
    \end{threeparttable}
\end{table}

The overall result is similar for both methods. A notable
difference is that there are more languages that have a positive effect size in
figure~\ref{fig:weffect_size} than in figure~\ref{fig:effect_size}. This means
that the most used words of the tested languages are more associated with
with word \emph{male} than all of the words in the language are.

\begin{figure*}[tb]%
    \caption{Figures \ref{fig:effect_size} and \ref{fig:weffect_size} show the
    measured effect size per language for both the uniform and the frequency weighting
    method respectively. Because these languages were compared against the dummy language
    described in section~\ref{seq:dummy_language}, the effect size is a metric of whether
words are, on average, more associated with \emph{male} or more associated with \emph{female}.}
    \label{fig:effect_weffect}
    \subfloat[Uniform weighting method. The effect size has been calculated with
    formula~\eqref{eq:weat_effect_size}. A value greater than zero
    indicates that words are on average more associated with \emph{male}, while a value
    below zero indicates that words are on average more associated with \emph{female}.]{
        \includegraphics[width=0.48\textwidth]{hist_effect_size.pdf}
        \label{fig:effect_size}
    } \quad
    \subfloat[Frequency weighting method. The effect size has been calculated with
    formula~\eqref{eq:frequency_effect_size}.
        A value greater than zero indicates that the most used words are more
        associated with \emph{male}, while a value below zero indicates that words are
        more associated with \emph{female}. Languages appended with two asterisk (**)
        were found to have \vbox{a \emph{p} value > $\alpha$.}]{
    \includegraphics[width=0.48\textwidth]{hist_weffect_size.pdf}
    \label{fig:weffect_size}
    }
\end{figure*}

\begin{figure*}[tb]%
    \caption{Figures \ref{fig:mean_difference} and \ref{fig:sum_weighted} show the mean
        and weighted mean of the cosine distance per language between each word and
        \emph{male} and each word and \emph{female} for the uniform and frequency
        weighting methods respectively.
    }
    \subfloat[Uniform weighting method. The mean of the cosine distances is
    calculated with formula~\ref{eq:uniform_s_mean}. A value greater than zero indicates 
        that a words are on average more
        associated with \emph{male}, while a value below zero indicates that words are on
        average more associated with \emph{female}.
    ]{
            \includegraphics[width=0.48\textwidth]{hist_bias.pdf}
    \label{fig:mean_difference}
        } \quad
    \subfloat[Frequency weighting method. The weighted mean of the cosine distances
    is calculated with formula \ref{eq:frequency_s_mean} and $f(\vec{x})$ as defined in
    section \ref{method:2}. 
        A value greater than zero indicates that a words are more
        associated with \emph{male}, while a value below zero indicates that words are
        more associated with \emph{female}.
        Languages appended with two asterisk (**)
        were found to have \vbox{a \emph{p} value > $\alpha$.} ]{
            \includegraphics[width=0.48\textwidth]{hist_wdiff.pdf}
        \label{fig:sum_weighted}
    }
\end{figure*}
